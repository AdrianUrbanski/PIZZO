\message{ !name(red.tex)}\documentclass{article}
\usepackage[polish]{babel}
\usepackage[T1]{ fontenc}
\usepackage[utf8]{ inputenc}
\usepackage{geometry}

\title{PIZZO - Zadanie domowe nr 1}
\geometry{margin=1in}
\author{Adrian Urbański}
\date{}

\begin{document}

\message{ !name(red.tex) !offset(-3) }

\maketitle
\paragraph{Podpunkt a)}
$3COL\leq_{p}Tutorzy$ \\

Pokażę, że $3COL\leq_{p}4COL$:\\
Zdefiniujmy funkcję $\Phi:G(V,E) \rightarrow G'(V',E')$ jako:\\
Jeśli $V = \{v_{1}, v_{2}, v_{3}, \cdots, v_{n}\}$, to $V' =
\{v_{1}^{1},v_{1}^{2},v_{2}^{1},v_{2}^{2},v_{3}^{1},v_{3}^{2}, \cdots,
v_{n}^{1}, v_{n}^{2}\}$. Wtedy:\\
$\forall i \in \{1,2,3,\cdots,n\}:  (v_{i}^{1},v_{i}^{2})\in E' \wedge \forall
(v_{i}, v_{j})\in E: [ (v_{i}^{1},v_{j}^{1})\in E' \wedge (v_{i}^{1},
v_{j}^{2})\in E' \wedge (v_{i}^{2}, v_{j}^{1})\in E' ]$. \\
Wtedy $"\Phi"$ jest redukcją wielomianową:
\begin{enumerate}
  \item $\Phi$ jest obliczalna w czasie wielomianowym - łatwo napisać program
    który tak przekształca graf
  \item G jest 3-kolorowalny wtw. gdy $\Phi(G)$ jest 4-kolorowalny
    \begin{itemize}
      \item[$"\Rightarrow"$]
        Weźmy 3-kolorowanie $C: V \rightarrow \{r, g, b\}$.\\
        Zdefiniujmy 4-kolorowanie $C': V' \rightarrow \{r, g, b, y\}$ takie, że:\\
        $C'(v_{i}^{1}) = r \Leftrightarrow C(V_{i}) = r $ i analogicznie dla g i b,
        oraz $\forall i: C'(v_{i}^{2}) = y$.\\
        Skoro C jest poprawnym kolorowaniem, to $\forall
        (v_{i}^{1},v_{j}^{1})\in E': C'(V_{i}^{1}) \neq C'(V_{j}^{1})$.\\
        W dodatku, $\forall i, j: (v_{i}^{2}, v_{j}^{2})\notin E'$, co wynika z
        konstrukcji $\Phi$, oraz $\forall i, j: C'(v_{i}^{2})=y\neq
        C'(v_{j}^{1})$ z definicji C'.\\
        Zatem $C'(\Phi(G))$ jest poprawnym kolorowaniem.
      \item[$"\Leftarrow"$]
        Weźmy 4-kolorowanie $C': V' \rightarrow \{r, g, b, y\}$. Wtedy:\\
        Niech
        $C(v_{i})=\left\{
        \begin{array}{ll}
          $r, gdy  $v_{i}^{1}=r \lor v_{i}^{2}=r\\
          $g, gdy  $v_{i}^{1}=g \lor v_{i}^{2}=g \land v_{i}^{1}\neq r \land
          v_{i}^{2}\neq r\\
          $b, wpp$
        \end{array}
        \right.$
        Wtedy:\\
        C jest 3-kolorowaniem, co wynika z konstrukcji $\Phi$
    \end{itemize}
\end{enumerate}

Pokażę, że $4COL\leq_{p}Tutorzy:$\\
Zdefiniujmy funkcję $\Psi:G(V,E) \rightarrow T(S,K)$, gdzie S jest zbiorem
studentów, a K - konflików ($(s_{i}, s_{j})\in K$ oznacza, że $s_{i}$ zrzędzi na
$s_{j}$), jako:\\
$\forall i: v_{i}\in V \Leftrightarrow s_{i}\in S \land \forall i, j: (v_{i},
v_{j})\in E \Leftrightarrow (s_{i}, s_{j})\in K$.\\ Wtedy $\Psi$ jest redukcją wielomianową:\\
\begin{enumerate}
  \item $\Psi$ jest obliczalna w czasie wielomianowym - łatwo napisać taki program
  \item G jest 4-kolorowalny wtw. gdy $\Psi(G)$ jest poprawnym przydzieleniem tutorów.
    \begin{itemize}
      \item[$"\Rightarrow"$]
        Weźmy 4-kolorowanie $C: V \rightarrow \{r, g, b, y\}$.\\
        Zdefiniujmy przydział tutorów $P: S \rightarrow \{t_{1}, t_{2},
        t_{3}, t_{4}\}$ takie, że:\\
        $P(s_{i}) = t_{1} \Leftrightarrow C(V_{i}) = r $ i analogicznie dla
        $t_{2}$ i g, $t_{3}$ i b oraz $t_{4}$ i y.\\
        Skoro C jest poprawnym kolorowaniem, to $\forall
        (s_{i},s_{j})\in K: P(s_{i}) \neq P(V_{j})$. Wtedy:\\
        Zatem $P(\Psi(G))$ jest poprawnym przydzieleniem tutorów.
      \item[$"\Leftarrow"$]
        Weźmy przydział tutorów P. Wtedy:\\
        Niech $C(v_{i})=\left\{
        \begin{array}{ll}$
          r, gdy $P(s_{i})=t_{1}\\$
          g, gdy $P(s_{i})=t_{2}\\$
          b, gdy $P(s_{i})=t_{3}\\$
          y, wpp $\\
          \end{array}
        \right.\\$
        Wtedy C jest 4-kolorowaniem, co wynika z konstrukcji $\Psi$.
    \end{itemize}
\end{enumerate}

Skoro $3COL\leq_{p}4COL \land 4COL\leq_{p}Tutorzy$, to z (przechodniości relacji
$\leq_{p}$) $3COL\leq_{p}Tutorzy$

\paragraph{Podpunkt b)} Jeśli jest co najwyżej 15 zrzęd, to problem Tutorzy jest
wielomianowy\\

Numerujemy tutorów od 1 do 4. Generujemy wszystkie możliwości (jest ich $4^{15}$) przydzielenia
tutorów zrzędom i dla każdej takiej możliwości dla każdego studenta (jest ich co
najwyżej n) sprawdzamy którego tutora możemy mu przydzielić (musimy sprawdzić wszystkie
osoby na które zrzędzono, jest ich co najwyżej $15*n$). Złożoność takiego
podejścia wynosi O($4^{15}*n*15*n$) = O($n^{2}$).
\end{document}

\message{ !name(red.tex) !offset(-107) }
